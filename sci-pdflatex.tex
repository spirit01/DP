%%%%%%%%%%%%%%%%%%%%%%%%%%%%%%%%%%%%%%%%%%%%%%%%%%%%%%%%%%%%%%%%%%%%
%% I, the copyright holder of this work, release this work into the
%% public domain. This applies worldwide. In some countries this may
%% not be legally possible; if so: I grant anyone the right to use
%% this work for any purpose, without any conditions, unless such
%% conditions are required by law.
%%%%%%%%%%%%%%%%%%%%%%%%%%%%%%%%%%%%%%%%%%%%%%%%%%%%%%%%%%%%%%%%%%%%

\documentclass[
  digital, %% This option enables the default options for the
           %% digital version of a document. Replace with `printed`
           %% to enable the default options for the printed version
           %% of a document.
  table,   %% Causes the coloring of tables. Replace with `notable`
           %% to restore plain tables.
  lof,     %% Prints the List of Figures. Replace with `nolof` to
           %% hide the List of Figures.
  lot,     %% Prints the List of Tables. Replace with `nolot` to
           %% hide the List of Tables.
  %% More options are listed in the user guide at
  %% <http://mirrors.ctan.org/macros/latex/contrib/fithesis/guide/mu/sci.pdf>.
]{fithesis3}
%% The following section sets up the locales used in the thesis.
\usepackage[resetfonts]{cmap} %% We need to load the T2A font encoding
\usepackage[T1,T2A]{fontenc}  %% to use the Cyrillic fonts with Russian texts.
\usepackage[
  main=english,  %% By using `czech` or `english` as the main locale
                %% instead of `slovak`, you can typeset the thesis
                %% in either Czech or English, respectively.
  english, german, russian, czech, slovak %% The additional keys allow
]{babel}        %% foreign texts to be typeset as follows:
%%
%%   \begin{otherlanguage}{german}  ... \end{otherlanguage}
%%   \begin{otherlanguage}{russian} ... \end{otherlanguage}
%%   \begin{otherlanguage}{czech}   ... \end{otherlanguage}
%%   \begin{otherlanguage}{slovak}  ... \end{otherlanguage}
%%
%% For non-Latin scripts, it may be necessary to load additional
%% fonts:
\usepackage{paratype}
\def\textrussian#1{{\usefont{T2A}{PTSerif-TLF}{m}{rm}#1}}
%%
%% The following section sets up the metadata of the thesis.
\thesissetup{
    date            = \the\year/\the\month/\the\day,
    university      = mu,
    faculty         = sci,
    department      = Ústav chemie,
    departmentEn    = Department of Chemistry,
    extra = {
      departmentCs  = Department of Chemistry,
    },
    programme       = Fyzikální chemie,
    programmeEn     = Physical Chemistry,
    extra = {
      programmeCs   = Chemistry,
    },
    field           = Fyzikální chemie,
    fieldEn         = Physical Chemistry,
    extra = {
      fieldCs       = Physical Chemistry,
    },
    type            = mgr,
    author          = Petra Hrozková,
    gender          = f,
    advisor         = doc. Markéta Munzarová Dr. rer. nat. ,
    title           = Studium elektronové struktury fosfosilikátů a jejich silikátových          prekurzorů metodou DFT,
    TeXtitle        = A DFT study of the Electronic Structure of Silicate Precursors for Phosphosilicates
.
,
    titleEn         = A DFT study of the Electronic Structure of Silicate Precursors for Phosphosilicates
,
    TeXtitleEn      = A DFT study of the Electronic Structure of Silicate Precursors for Phosphosilicates
,
    extra = {
      titleCs       =A DFT study of the Electronic Structure of Silicate Precursors for Phosphosilicates
,
      TeXtitleCs    = A DFT study of the Electronic Structure of Silicate Precursors for Phosphosilicates
,
    },
    keywords        = {kľúčové slovo 1, kľúčové slovo 2, ...},
    TeXkeywords     = {kľúčové slovo 1, kľúčové slovo 2, \ldots},
    keywordsEn      = {keyword1, keyword2, ...},
    TeXkeywordsEn   = {keyword1, keyword2, \ldots},
    extra = {
      keywordsCs    = {klíčové slovo 1, klíčové slovo 2, ...},
      TeXkeywordsCs = {klíčové slovo 1, klíčové slovo 2, \ldots},
    },
    abstract      = {This is the abstract of my thesis, which can

                     span multiple paragraphs.},
    abstractEn    = {This is the English abstract of my thesis, which can

                     span multiple paragraphs.},
    extra = {
      abstractCs   = {This is the Czech abstract of my thesis, which can

                      span multiple paragraphs.},
    },
    thanks        = {These are the acknowledgements for my thesis, which can

                     span multiple paragraphs.},
    bib           = example.bib,
    %% Uncomment the following line (by removing the % symbol at
    %% the beginning) and replace `assignment.pdf` with the
    %% filename of your scanned thesis assignment.
    % assignment    = assignment.pdf,
}
\usepackage{makeidx}      %% The `makeidx` package contains
\makeindex                %% helper commands for index typesetting.
%% These additional packages are used within the document:
\usepackage{paralist} %% Compact list environments
\usepackage{amsmath}  %% Mathematics
\usepackage{amsthm}
\usepackage{amsfonts}
\usepackage{url}      %% Hyperlinks
\usepackage{mhchem}
\usepackage{chemfig}
\usepackage{subfigure}
\usepackage{listings} %% Source code highlighting
\usepackage{braket}
%\renewcommand{\baselinestretch}{1.2}
%\usepackage{chemmacros}

\lstset{
  basicstyle      = \ttfamily,%
  identifierstyle = \color{black},%
  keywordstyle    = \color{blue},%
  keywordstyle    = {[2]\color{cyan}},%
  keywordstyle    = {[3]\color{olive}},%
  stringstyle     = \color{teal},%
  commentstyle    = \itshape\color{magenta}}
\usepackage{floatrow} %% Putting captions above tables
\floatsetup[table]{capposition=top}
\begin{document}




\chapter{Úvod}

\section{Experimentální motivace}
Cílem této práce je podat vysvětlení některých experimentálních jevů, které byly pozorovány u silikofosfátových polymerů. Studium vysoce porézních silikofosfátů je jedním ze zaměření skupiny anorganické a materiálové chemie na našem ústavu .
Studie prezentované v této práci se zaměřuje na na přítomnost hypervalentního pěti nebo šestikoordinovaného křemíku v jednotlivých strukturách. Postupně jsme hlavní otázku, řešenou v této práci, zformulovali do následující podoby: Jaká je souvislost mezi kombinací ligandů na čtyřkoordinovaném Si a Lewisovskou kyselostí těchto čtyřkoordinovaných sloučenin, vedoucí ke sklonu křemíku zvýšit svoji koordinaci na pět nebo šest ligandů. Modely reprezentují vybrané části z těchto polymerů a dělí se na dvě části. Anomální struktura silikofosfátových xerogelů neumožňuje jejich přímou strukturní charakterizaci. Modely navržené v této práci byly motivovány rentgenovými struktruami analogickýxh periodických struktur \cite{C3NJ00721A} anebo příbuznými strukturami \cite{Chipanina2011}(zde myslíte struktur s fluorem?)Dále byla využita data z NMR spektroskopie, z nichž byly odvozeny informace o výši koordinace křemíku a složení ligandů(fosfátové vs. organické estery) \cite{Styskalik2015thesis}.


\subsection{Silikofosfátové polymery}
Nejčastější formou výskytu křemíku v přírodě jsou křemičitany, sloučeniny obsahující křemík tetraedricky koordinovaný čtyřmi atomy kyslíku. Existují však i minerály s vyšší koordinací, např. thaumasit.[doplnit odkaz] Vzhledem k vysokému významu křemíku v přírodě jsou křemičitany rozsáhle připravovány a studovány i laboratorně. Oxid křemičitý \ce{SiO2} je po vodě nejvíce studovanou sloučeninou. Díky tomu, že \ce{Si^4+} je velikostně snadno zaměnitelný za \ce{Al^3+} nebo \ce{P^5+}, vznikají pak hlinitokřemičitany, obsahující Si-O-Al, (zeolity) nebo fosfokřemičitany, obsahující Si-O-P, tj. silikofosfáty. Obě skupiny molekul jsou rozsáhle studovány. Zmíněné skupiny jsou charakteristické svojí trojrozměrnou  porézní strukturou. Naše pozornost je v této práci soustředěna na silikofosfáty. Předpokládaná struktura připravených silikofosfátů studovaných v předchozích experimentálních prací je znázorněna na obrázku \ref{si_polymer_cely} \cite{Styskalik2015thesis}. Charakteristickou vlastnotstí struktur silikofosfátů je uspořádání jednotek Si-O-P do cyklů. Ačkoliv nejnovější struktura zahrnuje dvanáctičlenové cykly vzniklé trojnásobným opakování motivu Si-O-P-O (Pinkas et. aln Inorganic Chemistry), struktury s hypervalentním křemíkem obsahují osmičlenné cykly vzniklé dvojnásobným opakování jednotky Si-O-P-O[Pinkas]. Obrázek \ref{si_polymer_cely} znázorňuje předpokládanou strukturu silikofosfátového xerogleu se třemi druhy křemíkových center (koordinace čtyřmi fosfáty, šesti fosfáty, anebo čtyřmi fosfáty a dvěma organickými estery současně) a osmičlennými cykly Si-O-P. Stupeň koordinace křemíku a současně velikost pórů se ukázala být silně závislá na typu prekurzoru. Pokud byl ve výchozích sloučeninách jeden z fosfátů nahrazen methylovou skupinou přímo vázanou na křemík, ve výsledném xerogelu se nevyskytovay oktaedricky koordinované křemíky a velikost pórů byla větší. Silikofosfátové cykly jsou pak dále organizovány do vyšší stuktury skeletu mikroporézního (šířka pórů doo 2 nm) až mezoporézního (šířka póru 2-50 nm).
\begin{figure}[h!]
\caption{Silikofosfátová síť, \cite{Styskalik2015thesis}. }
  \center
  \includegraphics[width=12cm]{si_polymer_cely.png}
  \label{si_polymer_cely}
  \end{figure}
  Konkrétní metody příprav slikofosfátových sloučenin jsou uvedeny například v práci Aleše Stýskalíka  \cite{Styskalik2015thesis}.

\subsubsection{Křemík jako hypervalentní sloučenina}\label{teorie_hypervalence}
Podle klasické teorie AO mohou $p$ prvky tvořit čtyři vazby. Z experimentálních pozorování je ale známo, že prvky $p$  tvoří i více než čtyři vazby, obvykle pět nebo šest. Příklady?
Sloučeniny, kde se vyskytuje jeden nebo více atomů s více než osmi elektrony (oktet) se nazývají hypervalentní/hyperkoordinované. Sloučeniny zároveň musí být schopné vytvořit více vazeb než je číslo jejich atomových orbitalů. Konkrétně křemík může vytvářet 4-,5- a 6- koordinované sloučeniny a stát se hypervalentní.  \\
Existující, experimentálně připravené Hypervalentní sloučeniny s křemíkem lze rozdělit podle jednotlivých ligandů a jejich poloze v periodické tabulce. Křemík je schopen tvořit hypervalentní sloučeniny s fluorem, příkladem může být struktura  \ce{(SiF6)^{2-}} \ref{si_f6} \cite{memoriesphysiquelussac}. Tato struktura byla připravena v 19. století a považuje se za první připravenou sloučeninu křemíku v koordinaci šest. Pokud budeme postupovat ve skupine halogenů dolů, dalším ligandem by měl být logicky chlor. Sloučenina \ce{SiCl6^{2-}} není známá, naopak \ce{GeCl6^{2-}} ano. Schopnost atomu tvořit hypervalentní sloučeniny roste ve skupině dolů. Germanium má tedy vysokou schopnost tvořit hypervalentní sloučeniny, křemík ovšem potřebuje ligand s výrazně vyšší elektro elektronegativitou. Hypervalentní sloučeniny s chlorem byly proto připraveny až později, například \ref{si_cl_o} \cite{LAZAREV199716}.\\
Díky relativně vysoké elektronegativitě může fluor nahradit kyslík \ref{si_o_f}, \cite{C0DT01115K}.\\
Sloučenina, kde je křemík obklopen fluorem a dusíkem \ref{si_f_n}, \cite{C0DT01115K}.\\
Fluor s kyslíkem a vodíkem \ref{si_fluor_vodik_kyslik} \cite{BOYER19812165}.\\
Křemík s fluorem a uhlíkem \ref{si_with_fluor_carbons} \cite{kremikfluorcarbon}.\\
Křemík s fluorem a dusíkem \cite{si_f_n}.\\
Křemík pouze s kyslíkem v koordinačním okolí \ref{si_only_o}, cite{flyn1969}. Strukutra \ce{Sio6} se vyskytuje také v minerálu thaumasite \cite{Edge:a08100}.\\

Křemík s kyslíkem, uhlíkem a dusíkem \ref{si_n_o_c}, \cite{Wagler2014}.\\%TODO
Křemík s pouze s kyslíkem a dusíkem \ref{si_o_n} \cite{Wagler2014} .
Křemík pouze s uhlíke \ref{si_only_c}, \cite{A901953G}.
 \cite{Wagler2014}.
 \begin{figure}
 \begin{center}
   \subfigure[]{\includegraphics[width=2cm]{si_f6.png}\label{si_f6}}
 \subfigure[]{\includegraphics[width=3cm]{si_cl_o.png}\label{si_cl_o}}
 \subfigure[]{\includegraphics[width=5cm]{si_o_f.png}\label{si_o_f}}
 \subfigure[]{\includegraphics[width=3cm]{si_f_n.png}\label{si_f_n}}
 \subfigure[]{\includegraphics[width=5cm]{si_with_fluor_carbons.png} \label{si_with_fluor_carbons}}
 \subfigure[]{\includegraphics[width=5cm]{si_only_o.png} \label{si_only_o}}
 \subfigure[]{\includegraphics[width=5cm]{si_n_o_c.png}\label{si_n_o_c}}


 \subfigure[]{\includegraphics[width=5cm]{si_fluor_vodik_kyslik.png} \label{si_fluor_vodik_kyslik}}
 \subfigure[]{\includegraphics[width=5cm]{si_o_n.png} \label{si_o_n}}
 \subfigure[]{\includegraphics[width=5cm]{si_only_c.png} \label{si_only_c}}

 \label{obr_h4sio4_vysledky_I}



 \end{center}
 \end{figure}



\subsubsection{Fyzikální a chemické vlastnosti silikofosfátových sloučenin}
Samotné silikofosfátové polymery mají zajímavé fyzikální a chemické vlastnosti. Příkladem je Brønstedovská kyselost nebo vysoká protonová vodivost. JNěkolik z možných uplatnění silikofosfátových polymerů mohou být konduktory, elektrolyty, optická vlákna a biokompatibilní materiály. Fyzikální a chemické vlastnosti silikofosfátovýchh polymerů lze dobře vysvětlit pomocí analýzy molekulových orbitalů. Obecně mají Lewisovské kyseliny prázdné molekulové orbitaly, které leži dostatečně blízko obsazeným MO \footnote{MO = Molekulový orbital} konjugované báze. Vazba mezi křemíkem a jeho ligandem je silně ovlivněna elektronegativitou donorního ligandu. Atomy jako uhlík, dusík, kyslík, fluor nebo chlor podporují navyšování koordinace křemíku. V případě čtyřkoordinovaných sloučenin křemík poskytuje do vazeb všechny své valenční elektorny. Ve vyšším koordinačním stupeni už může křemík poskytnou pouze prázdné orbitaly a proto chová se jako Lewisovská kyselina.\\

Vazbu mezi křemíkem a kyslíkem lze považovat za dobrou Lewisovskou kyselinu. Z experimentu je známo, že \ce{SiO4} je dostatečnou Lewisovskou kyselinou, aby křemík mohl přímo reagovat s Lewivoskou bazí. Pokud je jeden z kyslíku ve struktuře nahrazen uhlíkem, schopnost navyšovat koordinaci je ztracena. Stejný jev pozoroval Aleš Sýskalík a spol. \cite{Styskalik2015thesis} a to vedlo k hypotéze o snížení Lewisovské kyselosti křemíku při tvorbě přímé vazby Si - C. Naopak pětikoordinovaný křemík je lepší Lewisovskou kyselinou než čtyřkoordinovaný a hypervalency podporuje.\cite{Wagler2014}\\

\begin{figure}[h!]
\caption{\cite{hypervalentsiliconmacmillangroup2005}}
  \center
  \includegraphics[width=12cm]{schema_silicophosphates.png}
  \label{schema_silicon_coordinate}
  \end{figure}

Pro vysvětlení hypervalence křemíku lze použít teorii hybridizace. Obecně se čtykoordinované sloučeniny vyskytují jako tetraedry, hybridizace $sp^3$. Pětikoordinované sloučeniny tvoří trigonální bipyramidu, hybridizace $sp^3d$. A šestikoordinované sloučeniny tvoří oktaedr, hybridizace $sp^3d^2$. (obrázek?) \\

Čtyřkoordinovaný křemík slňuje tetraedrické uspořádání. Při zvýšení koordinace na pět by měla být pozorována trigonální bipyramida, $sp^3d$. Výskyt $d$ orbitalu ve vazbě ale způsobuje nárust energie vazby na více než 200 kcal/mol. Z tohoto důvodu se předpokládá, že $d$ orbitaly se podílejí pouze na polarizaci porbitalů. Pentavalentní koordinace je poté realizována jako $3sp^2$ hybridizace doplněna třícenterní, čtyřelektronovou vazbou $3c-4e$ s p orbitalem.
 V případě pětikoordinované sloučeniny se ale spíše uvažuje hybridizace $sp^2$ a jedna vazba $3c-4e$ s p orbitalem, právě kvůli energii $d$ orbitalů. (obrázek?) \\
 Hypervalentní sloučeniny jsou lepší Lewisovské kyseliny díky d+ efektu na centralním křemíku. Důvodem je přesun elektronové hustoty na ligandy skrz nevazebné MO a podpora $3c-4e$ vazby. Rozložení elektronové hustoty molekulu stabilizuje a z tohoto důvodu se v hypervalentních sloučeninách vyskytují jako ligandy prvky s vysokou elektronegativitou. Tento jev dobře popisuje tzv. Benovo pravidlo:"Elektronegativní prvek dáva přednost vazbe s větším p-charakterem."\cite{hypervalentsiliconmacmillangroup2005}.\\
Pro křemík v koordinaci šest lze také předpokládat, že význam $d$ orbitalů nebude významný vzhledem k jejich energii. I zde se do vazby zapojí $3c-4e$ vazby\cite{Wagler2014}.\\

Další možnost interpretace hypervalence je založena na vysoké iontovosti vazby na křemíku. Obecně iontovost s koordinačním číslem roste.
Navíc chování vazby Si-ligand silně závisí na samotném ligandu a sterickém a elektronovém uspořádání. Hovoříme o Lewisovské kyselosti křemíkové vazby s elektronegativním atomem. Chování křemíku lze rozdělit na iontové, sigma vazebné a donor interakci.\cite{Wagler2014}\\

Z tohoto důvod jsme se rozhodli porovnávat Lewivoskou kyselost, abychom určili stabilitu jednotlivých částí silikofosfátů. Navíc jsme se snažili najít parametr, který by umožnil určit velikost póru v závislosti na okolí křemíku. Jako prostředek ke zkoumání silikofosfátů jsem zvolila molekulové orbitaly, které poskytují široké spektrum informací o molekule, vazbách, struktuře, kyselosti,..  Analýza byla provedena s pomocí teorie funkcionální hustoty(DFT)\footnote{DFT - Density Functional Theory, česky Teorie funkcionálu hustoty}, která se řadí mezi kvatově-chemické metody. Pro porovnání byla stejná analýza udělána s pomocí teorie Přirozených molekulových orbitalů(NBO)\footnote{NBO - Natural Bond Orbitals, česky Přirozené Molekulové orbitaly}. Výhoda přístupu NBO je snadnější převod čísel do chemického významu.


\section{Metody kvantové chemie}
Chemické vlastnosti molekul jsou určeny elektrony. Elektrony se řadí mezi elementární částice a proto pro jejich popis nelze použít běžnou newtonovskou mechaniku. Chování elektronů popisuje Schrödingerova rovnice, která se řadí mezi diferenciální rovnice druhého řádu. Její exatní řešení existuje pouze pro velice malé systémy, konkrétně systémy s dvěma částicemi. Z tohoto důvodu je v chemických aplikavích nutno použít zjednodušení. Základem je Born-Oppenhemrova aproximace(B-O) \ref{B_O_approximace}, která předpokládá, že pohyb elektronů a jader lze oddělit a Vlnová funkce elektronů závisí pouze na poloze jader, ne jejich rychlosti. Vlnová funkce se rozpadá na dvě části, elektronovou a jadernou. Řešení se redukuje na 3N prostorových souřadnic. Zároveň je vlnová funkce parametricky závislá na poloze jader, pro každou geometrii jader je získána jiná vlnová funkce. Ve druhém kroku dostaneme plochu potenciální energie (PES)\footnote{PES - Potential Energy Surface} pro každou geometrii a Schodingerova rovnice najednou závisí pouze na 3N prostorových souřadnících elektornů \cite{lechamolecularmodeling}.

 Dalšík krokem je Hartree-Fockova metoda (HF), někdy nazývána self-konzistentní metoda. HF využívá principů z Hartreeho metody, kdy se elektron pohybuje v časově průměrném poli ostatních elektronů, vlnová funkce je součin jednoelektronových funkcí. Výsledná vlvnová funkce je nalezena s pomocí variačního principu jako ta s nejnižší enenrgií. Orbitaly musí ortogonální a ortonormální.
 \begin{equation}
 S_{ii} = \int \psi_i * \psi_i dx dy dz = 1 ~ \wedge ~ S_{ij} = \int \psi_i * \psi_j dx dy dz = 0
 \end{equation}

\begin{equation}
  \Psi_{total} = \Psi_{electronic} \cdot \Psi_{nuclear}
  \label{B_O_approximace}
\end{equation}
Výsledkem je energie a uspořádání elektronů pro každou vzájemnou polohu jader. Původní Hartreeho metoda neobsahovala antisymetrii. Tu později doplnil Vladimir Aleksandrovič Fock a John Slater. Výsledkem je Hartree-Fockova metoda self-konzistentního pole(HF-SCF). Nejnižěí energie se hledá pomocí variačního počtu. Vlnovou funkci lze zapsat jako Slaterův determinant, který zaručuje antisymetrii vlnové funkce vůči výměně polohových a spinových souřadnic \ref{Slateruv_determinant}.
\begin{equation}
\psi =  \frac{1}{\sqrt{N!}}\begin{vmatrix}
\psi_1(1)\alpha(1) & \psi_1(1) \beta (1)  & \dots & \psi_{n/2}(1)\beta(1) \\
\psi_1(2)\alpha(2) & \psi_1(2) \beta (2) & \dots & \psi_{n/2}(2)\beta(2) \\
\vdots             & \vdots                           & \ddots & \vdots \\
\psi_1(n)\alpha(n) & \psi_1(n) \beta (n) & \dots & \psi_{n/2}(n)\beta(n)
\end{vmatrix}
\label{Slateruv_determinant}
\end{equation}
$\psi_i$ je jednoelektronová vlnová funkce, $s_j$ jsou elektrony, $\sqrt{N!}$ je normalizační faktor.
 Nevýhoda HF-SCF přístupu je fakt, že neuvažuje korelaci elektronového pohybu. Korelační energii lze vyjádřit jako rozdíl mezi přesnou nerelativistickou energii a HF limitou. Protože chemie uvažuje změny energie, tak je důležité korelační energiii vzít v úvahu.
Metody, které v sobě mají zahrnout korelační energii, nazýváme post HF metody. Poruchova teorie (MBPT)\footnote{MBPT - Many-Body Pertrubation theory}, metoda konfigurakční interakce (CI)\footnote{Configuration Interaction} nebo clustrové metody (CC)\footnote{Coupled
Cluster methods}. Časová složitost těchto metod je uvedena na obrázku. Velkou výhodou těchto metod je jejich přesnost, bohužel vzhledem k časové náročnosti je lze použít pouze pro malé systémy.


\section{Teorie funkcionálu hustoty}
Elektronová struktura každého chemického systému je popsána jako mnohoelektornová funkce. V HF přístupu není do výsledné energie zahrnuta korelační energie. Na druhou stranu metody, které korelační energii zahrnují mají dvě nevýhoy, výpočetní náročnost a složitost a lze je použít pouze pro malé systémy. Rozumným kompromisem může být přístup metodou funkcionálu hustoty. \\

Teorie funkcionálu hustoty\footnote{Z matematické analýzy je funkcionál operátor zobrazení z množiny funkcí do množiny obecně komplexních čísel.}poskytuje řešení pro velké systémy. Myšlenka DFT dává pohled na elektronovou strukturu z úplně jiného úhlu. Běžný postup hledání vlnové funkce je od znalosti externího potenciálu, přes Hamiltonián, vlnovou funkci až k energii. V DFT přístupu postupujeme přesně naopak. Ze znalosti elektronové hustoty k externímmu potenciálu, odsud k energii a všem vlastnostem systému. Elektronová hustota je pravděpodobnost, že v nějakém náhodném bodě v prostoru nalezeneme nějaký elektron. Velkou výhodou je fakt, že elektronová hustota je funkcí pouze tří prostorových souřadnic. \\
První DFT metody neuvažovaly žádné interakce, model elektronového plynu. Chyba v modelech dosahovala 15-50\%. V tomto modelu nelze mluvit o existenci molekul.\cite{jensen2007introduction}

Moderní DFT metody, které bylo možné aplikovat v chemii, se začaly objevovat po roce 1964 jako výsledek dvou teorémů. Trvalo to bezmála 40 let od modelu elektronového plynu. Prvním je Hohenberg-Kohn teorém (H-K), který mluví o základním stavu. Vnější potenciál  $V_ext$ je jednoznačnná funkce elektronové hustoty $\varrho$.
\begin{equation}
E = E_{el} (\varrho)
\end{equation}
 Všechny vlasnosti základního stavu mnohaelektronového systému jsou jednoznačně určeny elektronovou hustotou. Nevýhodou je, že elektronovou hustotou je určen pouze základní stav a pro excitované stavy jej nelze použít.

Samotná elektronová hustota je výsledkem druhého teorému. Ten říká, že nalezená elektornová hustota je jediná správná. Využívá variačního principu pro určení elektronové hustoty bez vlnové funkce. Správná elektronová hustota musí mít nejnižší energii a nejpřesnější energii.
\begin{equation}
E [\varrho_0] < E[\varrho ']
\end{equation}
 Energie excitovaného stavu musí být nižší než energie elektroové hustoty ostatních elektronů. Pro výpočetní chemii mají větší význam Kohn-Shamovy orbitaly, které byly výsledkem geniální myšlenky o rozdělení funkcionálu. Obecně je největší problém vyjádřit kinetickou energii elektornů. Ta se rozpadla na dvě části. Exaktní a korekční část.
\begin{equation}
E(\varrho(\vec{r})) = T_s[\varrho(\vec{r})] + J[\varrho(\vec{r})] + \int V_{EX}(\vec{r})\varrho(\vec{r})d\vec{r} + E_{XC}[\varrho(\vec{r})]
\end{equation}
 \cite{jensen2007introduction}\cite{koch2000chemist}
 \begin{equation}
 T_s = -\frac{1}{2} \sum_{i=1}^{N}  \bra{\psi_i}{\nabla^{2}}\ket{\psi_i}
 \label{kineticka_energie_jednoelektronova}
 \end{equation}
 Kinetická energie $T_s$ ale není přesným vyjádřením kinetické energie. Chybějící část je zajištěna $E_{EX}[\varrho(\vec{r})]$, výměnná-korelační energie. $J[\varrho(\vec{r})]$ je přesná Coulombovská repulze. $\int V_{EX}(\vec{r})\varrho(\vec{r})d\vec{r}$ je přesná energie atrakce elektronů jádry.  \cite{parr1994density}

\subsection{DFT metody v praxi}

\subsection{Bázové funkce}
Data z experimentu mohou pomoci s výběrem vhodné výpočetní metody. Jedním z přístupů je model bázových funkcí, kdy jsou MO hledány jako kombinace sady bázových funkcí. Podle postulátu QM o úplných vlastních hodnotách uplných Hermitovských operátorů lze každý molekulový orbital sestrojit jako lineární kombinaci atomových orbitalů, tzv. LCAO\footnote{LCAO - Linear Combination of Atomic Orbitals.}. Konkrétně jsou AO sestaveny jako kombinace orbitalů Gaussovského (GTO) nebo Slaterova typu(STO).
Kompletní sada bázovýchh funkcí obsahuje nekonečně mnoho funkcí, což dělá problém neřešitelný. Vhodný konečný počet funkcí byl určen s pomocí ladění chyb a poskytuje dostatečně přesné výsledky za příměřenou cenu. Každá elektronové korelačních metod škáluje rozdílně podle počtu elektronů(určeno počty obsazených MO) a podle sady bázových funkcí (počet neobsazených, virtuálních, MO).\\
Minimální sada bázových funkcí v sobě obsahuje elektrony v základním stavu. Naopak rozšířená sada bázových funkcí v sobě obsahuje polorizační funkce nebo difúzní funkce. Typ bázových funkcí mohou být vodíkového typu, GTO nebo STO. Nevýhoda funkcí vodíkového typu je délka výpočtu. STO oproti tomu nemají žádné radiální uzly a jejich sčítání není lineární. Vhodnou aproximací jsou orbitaly gaussovského typu, které sice nemají vhodné chování na jádře, ale platí pro ně pravidlo, že součet GTO je opět GTO.\cite{lowe2011quantum} GTO jsou využívány díky dobrým matematickým vlastnostem, i když STO nebo AO mají větší fyziální význam. V praxi se STO orbitaly vytvoří kombinací prmitivních gaussiánů.

Sada bázových funkcí přirozených atomových orbitalů je příkladem zmenšením báze. Přirozené orbitaly jsou takové, které diagonalizují matici hustoty a počet elektronů v orbitalu odpovídá obsazovacímu čísli.

\subsubsection{Pseudopotenciál (ECP)}
Systém s velkým počtem core, vnitřních, elektronů, jsou časové náročné. Často to jsou prvky za třetí periodou. Z pohledu chemické vazby je možno vliv core elektronů považovat za méně důležitý. Stejná situace nastává ve chvíli, kdy je v systému příliš mnoho elektornů. Jednoduše, systém je příliš velký. V obou případech lze použít funkci, ktra bude modelovat chování core elektronů. ECP má čtyři hlavní kroky. Prvním krokem je, že vlnová funkce pro všechny elektrony je vytvořdna s pomocí HF metody. Valenční orbitaly jsou nahrazeny skupinou psudoorbitalů bez nodálních ploch.

\section{Přirozené orbitaly (NBO)}
Přirozené molekulové orbitaly \footnote{NBO - Natural Bond orbital} jsopou odlišným pohledem na chemickou vazbu. Existuje spoustu přístupů k analýze vlnové funkce. Schrödingerova rovnice určuje energii pro každou vlnovou funkci. Vlnová funkce určuje pozici elektronů i jader. Jak je možné určite, zda jsou dva atomy spojené vazbou? Dobrým příkladem parametru, který lze použít pro určení vlasnosti molekul je atomový náboj. Obvyklé metody pro přiřazení náboje elektronu je analýza založená na bázových funkcí, elektrostatické potenciálu nebo vlnové funkci, lokalizovaných orbitalech nebo přirozených orbitalech. \\
První přístup používá MO a matici hustoty (DS)\footnote{DS - Density Matrix} a Mullikenovu populační analýzu. Čiste matematikcým přístupem je těžké určit správný výsledek. Sada populačních nábojů neodpovídá skutečným multipole moment.

Lepší přístup pro popis náboje je elektrostatický potenciál. Náboj je silně spojen s force field method. Nevazebné interakce josu popsány jako část elektrostatické interakce. Cčsitě matemtický přístup poskytuje metoda AIM\footnote{AIM = Atoms in Molecules}. Elektronová hustota může být vyhodnocena jako normální analytická funkce, která má své minimum, maximum a sedlové body. Hranice mezi dvěma atomy v tři dimenzionálním prostoru prostorem s dvěmi dimenzemi. \\
Dalším způsobem vyhodnocení jsou Lokalizované Orbitaly (NO).  První řád matice je diagonalizován a jeho vlastní vektory jsou označeny jako přirozené orbitaly. Vlastní hodnoty jsou obsazovací čísla. NO dávájí nejrychlejší konvergenci (Carlson-Keller teorem) a mohou popsat rozložení elektronů v atomech a odvození atomového náboje a vazby.  \cite{jensen2007introduction} \\
 NO jsou teoretickým zjednodušením, které určuje elektronovou hustoty v atomech a  tento přístup odpovídá přístupu Lewisovských struktur. NO byly objeveny v roce 1955 Per-Olov Lödwin, Löwdinův orthogonální algoritmus. NO mohou být vyvořeny z Slaterova determinantu. K-S nemají fyzikální význam a je zde problém s interpretací výsledků. Ale platí, že HF i K-S orbitaly mohou být použity pro tvorbu sady přirozených vazebných orbitalů. Přiozené orbitaly jsou vytvořeny diagonalizací matice hustoty
(NBO program)
 AOs -> NAOs -> NHOs -> NBOs -> NLMOs -> MOs

Lepší sada orbitalů se nazývá "přirozené" a jsou schopné zahrnout korelaci $\varrho$(r). Přirozené orbitaly mají maximální obsazenost a jsou určené ze samotné vlnové funkce.

\section{Tvrdost a měkkost kyselin a báze}
Chemickí tvrdost a měkkost má pro experimentální chemii velký význam . S pomocí analýzy tvrdosti/měkkost může být predikován produkt reakce a jeho stabilita. Bylo pozorováno, že reakce dvou tvrdých molekul poskytuje stabilní produkt, stejně to platí i pro reakci dvou měkkých molekul. \\
Globální tvrdost a měkkost můžou být použity pro celou molekulu. Jednotlivé atomy mají tvrdost a měkkost lokální. Tuto vlastnost lze interpretovat jako lokální náboj. Globální vlastnosti pochází z energie HOMO a LUMO orbitalů. Určení lokálních vlastností je obtížné, protože jsou spojeny s Fukuiho rovnicí. Ty popisují, který atom v molekule přijme nebo ztratí elektron. Chemická interpretace je schopnost nuklefilního nebo elektrofilního útoku. Stejně dobře to popíše externí elektrické pole a polarizace. Toto je přímo spojené s elektronovou hustotou a DFT metodami.

Elektrofilita atomu A v molekule M s N elektrony.
\begin{equation}
f_A^+ = P_A(N+1) - P_A(N)
\end{equation}
Nuklefilita atomu A v molekule M s N elektrony.
\begin{equation}
f_A^- = P-A(N) - P_A(N-1)
\end{equation}
Radical attack susceptibility of atom A in molecule M with N electrons.
\begin{equation}
f_A^0 = \frac{1}{2}[P_(N+1) - P_A(N-1)]
\end{equation}
Nalezení obsazovacího čísla na každém atommu je citlivé na výběru baze. Příliš velká baze dává špatné výsledky.

\section{Mullikenova populační analýza}

\section{Výpočetní detaily}


\section{Výpočetní část}
Výpočetní část byla rozdělena do tří částí. První část se zabývá stukturami, druhá část dává podrobnější pohled na vazby ve zvolených strukturách a třetí, poslední, část se zaměřuje na spektroskopii.
Cílem strukturní části bylo modelovat menší části struktur silikofosfátů metodami DFT. Modely silikofosfátových polemyrů byly rozděleny do tři skupin podle velikosti a kooridnace křemíku. Vazby v modelech silikofosfátů byly analyzovány metodou NBO a Mullikenovou populační analýzou. Poslední část, která se věnuje spektroskopii, dává pohled na NMR parametry křemíku a porovnává je s experimentálně získanými hodnotami.

\subsection{Struktury}
Struktury pro teoretickou část jsou rozděleny podle velikosti, podle stupně koordinace křemíku a podle množství cyklů ve strukturách. Zvolené modely byly tvořeny podle experimentální motivace. Snahou bylo co nejlépe modelovat koordinační okolí křemíku. Malé sturktury měly za cíl podrobnou analýzu vazby křemík-uhlík. Jednoduchý model umožňoval detailní pohled na chemickou vazbu.\\

Modely střední velikosti reprezentovaly širší okolí křemíku. (acetoxymethyl)trifluorosilan sloužil jako model křemíku, kde se vyskytovala příma vazba na uhlík a zároveň byly kolem vysoce elektronegativní fluory. Jak bylo zmíněno v kapitole \ref{teorie_hypervalence}, křemík je ochotný navyšovat koordinaci s vysoce elektronegativními ligandy. Zároveň je (acetoxymethyl)trifluorosilan schopen tvořit intramolekulární vazbu s Si-O a tím navbýšit koordinaci křemíku na pět. Tyto strukutry jsou experimentálně pozorovány a nazývají se dragonoidy \cite{Chipanina2011}. Další dva střední modely už obsahovaly cykly, prozatím ale malé. \\
Model middle1 \ce{SiCH3(PO4)CH3(SiP2O10)(CH3)4} obsahoval jeden cyklus, volně navázaný fosfát a přímou vazbu křemík-uhlík. Model middle2 \ce{Si(P2SiO10(CH3)4)2} už obsahoval dva cykly, které byly pozorovány v silikofosfátových polymerech. Jednalo se o nejmenší model, který už obsahoval dva cykly.\\

Největším problémem se ukázaly velké struktury, kde nalezení vhodného modelu trvalo nejdelší dobu. Nalezení vhodného modelu se stalo klíčovým problémem této práce.  Výchozí strukturou byla struktura z rentgenové analýzy \cite{C3NJ00721A}, kde se křemík vyskytoval v koordinaci šest. Z tohoto modelu vycházely všechny ostatní struktury se šestikoordinovaným křemíkem. V experimentálních struktruách silikofosfátů se krom fosfátových skupiny vyskytovaly tak různé acetylové skupiny. Právě přítomnost acetylových skupin měla vliv na chování křemíku. Acetylové skupiny  se v okolí křemíku vždy vyskytovaly po dvou a toto bylo dodrženo i v modleových strukturách. Zvolila jsem dvě možnosti umístění acetylových skupin, poloha cis a trans(obrázek). První návrh byl velice podobný původní krystalové struktuře, včetně přítomnost osmi cyklů. To se ukázalo jako problematické a pro různé modely bylo nutno dělat rozdílná uvolnění cyklů. Zde je klíčové zminit, že naše modely byly pouze výseky z rozsáhlých struktur a chyběla stabilizace okolní hmotou. Pro všechny modelové struktury bylo nutno povolít napětí mezi cykly. Původní krystalová struktura obsahovala šest cyklů. Mnou upravené struktury obsahovaly cykly dva (cis) nebo čyři(trans). Další koordinací, která byla pozorována byl křemík v koordinaci pět. Tento typ nebyl příliš častý, ale řadí se mezi experimentálně pozorované. Proto jsem také pětikoordinovaný křemík zařadila do analýzy. V případě pětikoordinované struktury byla snaha kompenzovat náboj a dosáhnout náboje nula, stejně jako u předchozích struktur. To se nakonec ukázalo jako nevhodné a model s křemíkem v koordinaci pět měl náboj mínus jedna.

\subsection{Čtyřkoordinovaný křemík jako reaktant}
Na základě experimentální práce Aleše Stýskalíka jsem se zabývala křemíkem v koordinaci čtyři, který sloužil jako výchozí reaktant pro přípravu silikofosfátových polymerů.

\subsubsection{Malé modely}
Malé modely podávají podrobnější pohled na charakter vazby Si-C. Obrázky struktur jsou uveden na obrázku \ref{prehled_male_modely}. nejmenších problémů. Pro struktura \ref{si_ch3_och3_5} bylo nutno použit Opt=CalcFC. Pro sturkturu \ref{si_och3_6} byl použit parametr SCF=Vshift. Pro vybrané modely byla provedena analýza pomocí NBO a MPA. Výsledky jsou uvedeny v tabulce.
\begin{figure}
\begin{center}
\subfigure[]{\includegraphics[width=5cm]{si_ch3_och3.png} \label{si_ch3_och3}}
\subfigure[]{\includegraphics[width=5cm]{si_och3_4.png}\label{si_och3_4}}
\subfigure[]{\includegraphics[width=5cm]{si_ch3_och3_5.png}\label{si_ch3_och3_5}}
\subfigure[]{\includegraphics[width=5cm]{si_och3_6.png}\label{si_och3_6}}
\caption{Přehled malých struktur.}
\label{prehled_male_modely}

\end{center}
\end{figure}


\subsubsection{Středně velké modely}
Tato část se věňuje nejmenším možným modelům, které již tvoří uvnitř svých struktur cyklus. Jako referenční molekula byl použita \ce{(acetylmethoxyl)trifluorsilan} z článku \ref{Chipanina2011}. Strutktura \ce{(acetylmethoxyl)trifluorsilan} slouží jako model křemíku v koordinaci čtyři, který má ve svém okolí vysoce elektronegativní atomy. To podporuje teoretické předpoklady o hypervalenci křemíku, pokud má ve svém okolí silně elektronegativní prvek.
\begin{figure}
\begin{center}
\subfigure[(acetylmethoxyl)trifluorsilan]{\includegraphics[width=5cm]{acetylmethyltrifluor.png} \label{obr_h4sio4_MO_s1_1}}
\subfigure[]{\includegraphics[width=5cm]{si_model_methyl.png}\label{obr_h4sio4_MO_s1_20}}
\subfigure[]{\includegraphics[width=5cm]{si_model_orezany.png}\label{obr_h4sio4_MO_s1_24}}
\caption{}
\label{obr_h4sio4_vysledky_I}
\end{center}
\end{figure}
Pro zvolené sloučeniny byla provedena analýza kanonických orbitalů, výsledky jsou uvedeny v tabulce.
\subsubsection{Velké modely}
\begin{figure}
\begin{center}
\subfigure[(acetylmethoxyl)trifluorsilan]{\includegraphics[width=5cm]{struktura_cis.png} \label{obr_h4sio4_MO_s1_1}}
\subfigure[]{\includegraphics[width=5cm]{srtuktura_bez_naboje.png}\label{obr_h4sio4_MO_s1_20}}
\subfigure[]{\includegraphics[width=5cm]{struktura_puvodni.png}\label{obr_h4sio4_MO_s1_24}}
\subfigure[]{\includegraphics[width=5cm]{struktura_trans.png}\label{obr_h4sio4_MO_s1_24}}
\label{obr_h4sio4_vysledky_I}



\end{center}
\end{figure}



\subsection{Vazby}

\subsection{Spektroskopi}














{\csname captions\languagename\endcsname %% Temporarily override
%% the BibLaTeX localization with the original babel definitions.
\makeatletter %% Use the correct localization of the quotations.
  \thesis@selectLocale{\thesis@locale}\makeatother
\printbibliography[heading=bibintoc]} %% Print the bibliography.
\appendix %% Start the appendices.




\end{document}
